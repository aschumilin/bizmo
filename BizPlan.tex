\documentclass[12pt, a4paper]{article} % Document font size and paper size
\usepackage[utf8]{inputenc}

\usepackage{tgpagella}
\usepackage[T1]{fontenc}

\usepackage{hyperref}
\usepackage{geometry} % Allows the configuration of document margins
\geometry{a4paper, textwidth=6in, textheight=8.5in, marginparsep=6pt, marginparwidth=.6in} % Document margin settings
\usepackage{color}
\usepackage{listings}
\lstset{literate=%
    {Ö}{{\"O}}1
    {Ä}{{\"A}}1
    {Ü}{{\"U}}1
    {ß}{{\ss}}2
    {ü}{{\"u}}1
    {ä}{{\"a}}1
    {ö}{{\"o}}1
}


\begin{document}

%%%%%%%%%%%%%%%%%%%%%%%%%
% TITEL
%%%%%%%%%%%%%%%%%%%%%%%%%
\thispagestyle{empty}

\begin{titlepage}
%%\let\footnotesize\small \let\footnoterule\relax
\begin{center}
\hbox{}
\vfill

{\color{blue}\huge\bfseries Semantic Information Firewall\par}
\vskip 1.8cm
Business Plan\\
von\\[2mm]
\vskip 1cm

{\large\bfseries Artem Schumilin \& Igor Tseyzer\\}
\vskip 4cm
{\bfseries Seminar}\\
Developing Business Models \\
for the Semantic Web (WS 13/14) \\
%Universität Karlsruhe (TH)\\[2ex]
\vskip 3cm

\vskip 3cm
December 2013

\end{center}
\vfill
\end{titlepage}

%%%%%%%%%%%%%%%%%%%%%%%%%
% CONTENT
%%%%%%%%%%%%%%%%%%%%%%%%%

\tableofcontents

\newpage

%%%%%%%%%%%%%%%%%%%%%%%%%
% Executive Summary Artem und Igor
%%%%%%%%%%%%%%%%%%%%%%%%%

\section{Zusammenfassung}

Semantic Information Firewall ist ein präzise und zuverlässige Web-Dienstleistung, die für Informationssuche in Bereich Informatik entwickelt. Auf Basis von semantische Quellenindexierung erlaubt Information Firewall real-time Suche in Informationquellen, Informationsabfrageverwaltung und benachbarte Informationsquellensuche.\\
Die Zielgruppe ist vor allem Studenten und Wissenschaftler in Bereich Informatik, Hochschulen, IT-Unternehmen und alle, die in Bereich Informatik arbeiten und forschen.\\
Das Projekt wurde von zwei Studenten von Karlsruher Institut für Technologie gegründet. Das Team hast entsprechende Kenntnisse in Bereichen von Web Technologies, Semantic Technologies, Softwareentwicklung sowie Unternehemens- und Finanzverwaltung. \\


\emph{Fassen Sie Ihr Vorhaben auf höchstens zwei Seiten zusammen. Informieren Sie über Ihr Produkt oder Ihre Dienstleistung, Ihre Zielgruppe, Ihre Vorkenntnisse in der Branche, Ihre Personal- und Finanzplanung und darüber, welchen Umsatz und Gewinn Sie in den ersten Jahren anstreben. Halten Sie die Zusammenfassung bewusst kurz und knapp und beschreiben Sie Ihre Idee so, dass auch Branchenfremde Ihr Vorhaben verstehen. Tipp: Schreiben Sie die Zusammenfassung erst, wenn Sie alle anderen Punkte Ihres Businessplans ausgearbeitet haben. So wird es Ihnen leichterfallen, Ihr Vorhaben in der nötigen Kürze zu erläutern.}

\newpage


%%%%%%%%%%%%%%%%%%%%%%%%%
% Geschäftsidee Artem
%%%%%%%%%%%%%%%%%%%%%%%%%
\section{Geschäftsidee}

Unter Punkt 2 können Sie ausführlicher auf Ihr Vorhaben eingehen. Folgende Fragen können als Leitfaden dienen: Wie schätzen Sie die Erfolgsfaktoren für Ihr Angebot ein? Welche Marktstrategie wollen Sie verfolgen? Auch Angaben zum geplanten Firmennamen, zum Standort, zur Rechtsform und zur Firmenstruktur gehören in diesen Abschnitt. Darüber hinaus sollten Sie an dieser Stelle Ihre Unternehmensziele für die nächsten drei bis fünf Jahre formulieren.



\subsection{Gründungsvorgeschichte}

- Urheber der Geschäftsidee und vorhandene Schutzrechte sowie Verknüpfung mit vorhergehenden Projekten
- Einbindung des Gründungsvorhabens in das Umfeld der Hochschule bzw. der Forschungseinrichtung 

\subsection{Know-how Träger}
- kurze Vorstellung des geplanten Gründungsteams und Aufgabenverteilung der beteiligten Personen
- Bedeutung des an der Hochschule bzw. der Forschungseinrichtung erworbenen Know-hows für das   
  Gründungsvorhaben
- vorhandene betriebswirtschaftliche bzw. unternehmerische Erfahrungen bzw. Ausbildungen
- Einbindung weiterer wissenschaftlicher Berater, Partner, Mentoren etc.
\subsection{Innovation}
- Beschreibung der Technologie- oder Produktinnovation bzw. der wissensbasierten innovativen Dienstleistung
- Entwicklungsansatz und derzeitiger Stand der Umsetzung (ggf. Labor- oder Funktionsmuster)

\subsection{Projektplanung}
- projektbezogener Arbeitsplan für den Förderzeitraum und Ausblick auf Aktivitäten bis zur Marktreife
\newpage


%%%%%%%%%%%%%%%%%%%%%%%%%
% Analyse des Marktes Igor
%%%%%%%%%%%%%%%%%%%%%%%%%
\section{Analyse des Marktes}

Wer eine Firma gründen will, sollte die Branche und den Markt gut kennen. Wer ist die Zielgruppe? Wie viel Einkommen haben Ihre möglichen Kunden? Nicht unwichtig ist auch: Wie steht es um die Zahlungsmoral der Kunden? Bei all diesen Fragen sollten Sie im Blick behalten, ob Sie Ihre Produkte nur regional, bundesweit oder sogar international anbieten wollen. Auch ein Blick auf die Wettbewerber ist wichtig: Wer sind sie und welche Strategien verfolgen sie? Gibt es aktuelle Trends, die sich auf Ihr Geschäft auswirken könnten – wie etwa eine verstärkte Nachfrage nach Produkten, die Sie anbieten wollen?

\subsection{Marktsituation}
- Daten zu Marktvolumen, Marktsegmenten, prognostizierten Marktwachstum und Marktpotenzial 

\subsection{Alleinstellungsmerkmal und Kundennutzen}
- Alleinstellungsmerkmal des Produkts oder der Dienstleistung („einzigartige Verkaufseigenschaft“ oder auch USP = Unique Selling Proposition genannt) gegenüber Konkurrenten mit vergleichbarem Portfolio
- Kundennutzen des Produkts bzw. der Dienstleistung

\subsection{Wettbewerber}
- Aufzählung der wesentlichen Wettbewerber und Abgrenzung gegenüber deren Angeboten 

\subsection{Markteintritt}
- Angaben zur Zielgruppe und potenziellen Kunden (Pilotkunden vorhanden?)
- mögliche Markteintrittsbarrieren und Maßnahmen zu Marketing und Vertrieb
- strategische Partnerschaften beim Markteintritt



\newpage


%%%%%%%%%%%%%%%%%%%%%%%%%
% Marketingstrategien Igor
%%%%%%%%%%%%%%%%%%%%%%%%%
\section{Marketingstrategien}

Die Dienstleistung wird in drei Wichtige Formen für Kunden dargestellt:
\begin{itemize}
\item Webauftritt
\item App für Mobile Geräte
\item RSS Feed
\end{itemize}

Das Geldeinfluss wird von drei Perspektiven Betrachten. Erste ist die direkte Dienstleistung für Kunden. Sie konnen aus drei Varianten raussuchen, die in Tabelle \ref{tab:kunden} dargestellt

\begin{table}[h]
  \centering
  \begin{tabular}{|l|l|}\hline
  Dienstleistung & Beschreibung \\ \hline
  Microtransactions & Kunden haben eine Möglichkeit nur für Abgefragte Information zu zahlen \\ \hline
  Subscriptions & Kunden zahlen feste Preis per Zeiteinheit für Dienstleistungsnutzung. Kunden können von zwei Varianten aussuchen - Basic und Professional \\ \hline
  Freemium & Kostenlose Zugriff zur Dienstleistung für Kunden, die für ein Monat Begrenzt \\ \hline
  \end{tabular}
  \label{tab:kunden}
\end{table}  

Nutzung von Webauftritt wird auf folgende Bedingungen dargestellt:
\begin{itemize}
\item Microtransactions: Kunden haben eine Möglichkeit nur für Abgefragte Information zu zahlen. 

\item Subscriptions: Kunden zahlen feste Preis per Zeiteinheit für Dienstleistungsnutzung. Kunden können von zwei Varianten aussuchen - Basic und Professional.

\item Freemium: Kostenlose Zugriff zur Dienstleistung für Kunden, die für ein Monat Begrenzt.
\end{itemize}



\emph{Daher beschreiben Sie an dieser Stelle, wie Sie auf Ihr Unternehmen aufmerksam machen wollen. Stellen Sie einen detaillierten Zeitplan auf, zu welchem Zeitpunkt Sie welche Schritte gehen wollen: Wann wollen Sie Ihr Geschäft eröffnen? Wann und wie wollen Sie die ersten Kunden gewinnen? Denken Sie auch an Werbung und PR: Wann gehen Sie mit der Vorstellung Ihres Geschäfts an die Presse? Wie viel Geld haben Sie für Werbung im Budget? Außerdem erläutern Sie hier den Vertriebskanal, über den Sie Ihr Produkt verkaufen wollen. Nicht vergessen: eine Beschreibung Ihrer geplanten Preisgestaltung.}


\newpage

%%%%%%%%%%%%%%%%%%%%%%%%%
% Unternehmensplanung Igor
%%%%%%%%%%%%%%%%%%%%%%%%%

\section{Unternehmensplanung}
- geplante Rechtsform und Organisation bzw. Organigramm für das zu gründende Unternehmen

Der erste Mitarbeiter in Ihrem Unternehmen sind zunächst Sie selbst: Beschreiben Sie daher, welche beruflichen Erfahrungen Sie in der Branche gesammelt haben und wie Ihre bisherigen Erfolge aussahen. Haben Sie vielleicht eine Weiterbildung gemacht oder sich auf anderem Wege wertvolles Wissen angeeignet? Wie sieht Ihre Personalplanung für die nächsten drei bis fünf Jahre aus: Wollen Sie überhaupt Mitarbeiter einstellen und wenn ja, wie viele? Wollen Sie vielleicht mit freien Mitarbeitern oder Aushilfen zusammenarbeiten? So wie Sie Ihre eigenen Kenntnisse aufgelistet haben, sollten Sie auch die Qualifikationen Ihrer möglichen Partner und der schon feststehenden Mitarbeiter beschreiben.

\newpage

%%%%%%%%%%%%%%%%%%%%%%%%%
% Chancen und Risiken Artem
%%%%%%%%%%%%%%%%%%%%%%%%%
\section{Chancen und Risiken}
- mögliche Schwierigkeiten bei der Entwicklung des Produkts bzw. der Dienstleistung, bei der Gewinnung von Investoren/Geldgebern sowie von Mitarbeitern oder beim Markteintritt 

Dieser Part darf in keinem Businessplan fehlen. Um zu zeigen, dass Sie Ihr Vorhaben realistisch sehen, beschreiben Sie an dieser Stelle, welche Risiken und welche Chancen auf Sie zukommen könnten. Wie wollen Sie auf mögliche Veränderungen am Markt reagieren? Wie könnten sich positive und negative Ereignisse auf Ihr Unternehmen auswirken? Wer beweist, dass er auch das Risiko mit einplant, zeigt seinen Geschäftspartnern, dass er verantwortungsbewusst gründen will. Banken sehen außerdem gern ein Best-Case- und ein Worst-Case-Szenario: Was passiert, wenn sich alle Ihre Erwartungen erfüllen? Und wie könnte die Geschäftsentwicklung im ungünstigsten Fall aussehen? Belegen Sie Ihre Angaben möglichst mit überprüfbaren Zahlen und Fakten.

\newpage

%%%%%%%%%%%%%%%%%%%%%%%%%
% Finanzplanung Artem
%%%%%%%%%%%%%%%%%%%%%%%%%
\section{Finanzplanung}
- Einnahmen- und Ausgabenplanung für die ersten zwei Jahre nach Gründung mit Erläuterungen 

Geld spielt bei der Existenzgründung eine große Rolle. Daher sollten Sie in Ihrem Businessplan diesen Punkt exakt ausarbeiten und mit Zahlen und Fakten belegen. In Teil 1 nennen Sie die geplanten Investitionen, die Sie am Anfang und im Laufe der ersten Zeit tätigen. Listen Sie alle größeren Anschaffungen der nächsten drei bis fünf Jahre auf. Aus dieser Liste ergeben sich der Kapitalbedarf und die jährlichen Abschreibungen. Teil 2 ist der Liquiditätsplan: Er zeigt alle Ausgaben und Einnahmen, die Sie in den nächsten Jahren erwarten. Für das erste Jahr machen Sie eine monatliche Aufstellung, für die Folgejahre reicht eine Auflistung zunächst nach Quartalen, später pro Halbjahr. Darüber hinaus erstellen Sie eine Gewinn- und Verlustrechnung über den voraussichtlichen Geschäftsverlauf samt Umsätzen, Ausgaben und Gewinnen. Anschließend ordnen Sie die Finanzierungsposten den Kapitalgebern zu: Wie viel Eigenkapital haben Sie, und wofür setzen Sie es ein? Und wie viel Unterstützung durch Banken oder andere Kreditgeber brauchen Sie für welche Posten? Vergessen Sie auch nicht eine Reserve für unvorhersehbare Ausgaben.



\newpage


\end{document}

\section{Finanzplanung}



- Einnahmen- und Ausgabenplanung für die ersten zwei Jahre nach Gründung mit Erläuterungen 

Geld spielt bei der Existenzgründung eine große Rolle. Daher sollten Sie in Ihrem Businessplan diesen Punkt exakt ausarbeiten und mit Zahlen und Fakten belegen. In Teil 1 nennen Sie die geplanten Investitionen, die Sie am Anfang und im Laufe der ersten Zeit tätigen. Listen Sie alle größeren Anschaffungen der nächsten drei bis fünf Jahre auf. Aus dieser Liste ergeben sich der Kapitalbedarf und die jährlichen Abschreibungen. Teil 2 ist der Liquiditätsplan: Er zeigt alle Ausgaben und Einnahmen, die Sie in den nächsten Jahren erwarten. Für das erste Jahr machen Sie eine monatliche Aufstellung, für die Folgejahre reicht eine Auflistung zunächst nach Quartalen, später pro Halbjahr. Darüber hinaus erstellen Sie eine Gewinn- und Verlustrechnung über den voraussichtlichen Geschäftsverlauf samt Umsätzen, Ausgaben und Gewinnen. Anschließend ordnen Sie die Finanzierungsposten den Kapitalgebern zu: Wie viel Eigenkapital haben Sie, und wofür setzen Sie es ein? Und wie viel Unterstützung durch Banken oder andere Kreditgeber brauchen Sie für welche Posten? Vergessen Sie auch nicht eine Reserve für unvorhersehbare Ausgaben.


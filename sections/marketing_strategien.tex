\section{Marketingstrategien}



\subsection{Dienstleistungsformen}

Die Dienstleistung stellt Zugangsformen für Kunden dar:
\begin{itemize}
\item Webauftritt
\item App für Mobile Geräte
\item RSS Feed
\end{itemize}

Die Dienstleistung wird wird vor allem als Webauftritt dargestellt. App für Mobile Geräte und RSS Feed sind für Unterstützung der Webauftritt und Erweiterung der Nutzbarkeit und Nutzungserfahrung genutzt.\\

Die Dienstleitung stellt folgende Funktionen zur Verfügung:
\begin{itemize}
\item Sematische Suche der Publikationen
\item Suche in verwandte Felder. 
\item Verwaltung und Auswahl der wissenschaftlicher Suchquellen
\item Gestaltung von eigene Publikationslisten und Share-Funktion.
\item Gestaltung von Literaturverzeichnis für eigene Publikationen und Export-Funktionen in {\LaTeX} und andere Formaten. 
\item Verwaltung genutzter Publikationen
\item Self-Learning Algorithmus zur Verbesserung Suchergebnisse, basiert auf vorher gesuchte Publicationen.
\end{itemize}

Dieser Funktionen sind Anfangsfunktionen und werden durch Kunden-Feedback aktualisiert. 

\subsection{Nutzungsverfahren der Dienstleistung}
Die Dienstleistung für Endnutzer wird auf folgende Bedingungen angeboten:
\begin{itemize}
\item Funktion der Sematische Suche wird auf Webauftritt kostenlos und ohne Anmeldung angeboten
\item Zugang zur andere Funktionen der Dienstleistung erfolgt nur nach der Anmeldung und entsprechend der ausgewählte Nutzungsform. 
\end{itemize}
Dieser Gliederung dient für Kundengewinnung und stellt eine Möglichkeit die Dienstleistung unverbindlich zur nutzen. Die genaue Nutzungsformen werden in der Abschnitt Monetarisierung dargestellt. 

\subsection{Monetarisierung}

Die Monetarisierung wird von zwei Perspektiven Betrachten - direkt und indirekt. Bei direkte Monetarisierung, Geldeinfluss kommt gerade von Kunden. Die Kunden können die Dienstleistung in zwei Formen nutzen:
\begin{itemize}
\item \textbf{Microtransactions}. Kunden haben eine Möglichkeit nur für Abgefragte Information zu zahlen. 

\item \textbf{Abonnement}. Kunden zahlen feste Preis per Monate für Dienstleistungsnutzung, die in zwei Varianten gegliedert - Basic und Professional.
\end{itemize}

Microtransactions sind gut für private Kunden geeignet, die Information nicht ständig brauchen. Dieser Form erlaubt einmalige Nutzung der verfügbare Funktionen pro eine Zahlung. Abonnement ist gut für Wissenschaftliche Einrichtungen und Unternehmen geeignet. Man kann von zwei Zugangsniveau wählen. Basic stellt eine Möglichkeit kostengünstig Funktionen "Suche in verwandte Felder" und "Verwaltung und Auswahl der wissenschaftlicher Suchquellen" nutzen. Bei Professional, kann man alle existierende Funktionen unbegrenzt nutzen.\\

Bei indirekte Ansicht wird das Geldeinfluss aus Quellen, die nicht bei Endnutzer sind. Dieser Quellen sind
\begin{itemize}
\item \textbf{Sponsorships}. Geldeinfluss von Unternehmen oder Einrichtungen, die Bekanntheit zu erhöhen oder naher Kommunikation mit bestimmte Kundengruppe gestalten wollen.
\item \textbf{Werbung}. Kontext-basierte Werbung auf Webauftritt.
\end{itemize}

Zusammen direkte und indirekte Monetarisierung stellen eine Möglichkeit eine glatte Geldeinfluss zu gestalten und am schwierige Anfangsphase Finanziell gesichert sein.

 
\subsection{Kommunikationspolitik}

Für eine erfolgreiche Markteintritt und Absatz braucht man eine Kommunikationspolitik mit Kunden festzustellen und entsprechend nutzen. Um die Dienstleistung für potenzielle Kunden bekannt zumachen und eine Möglichkeit zu gestalten die Dienstleistung zu nutzen, werden folgende Maßnahmen durchgeführt: 
\begin{itemize}
\item Freemium: 3 Monate von kostenfreie Nutzung der Professional-Version für Kunden, die Basic-Version gekauft haben.
\item 3 Monate kostenlose Zugang zur Basic-Version, den bei Partnern ausgegeben wird (Webseiten, Zeitschriften etc.)
\item Speziale Angebote für wissenschaftliche Einrichtungen (Rabatte oder kostenlose Zugang)
\end{itemize}

\subsection{Markteintrittplan}

Markteintritt kann in vier Phasen gegliedert werden. Alle Phasen sollen Erfolg der Dienstleistung auf den Markt sichern und unterstützen: 
\begin{enumerate} 
\item Vorbereitung
\item Eintritt 
\item Kontrolle 
\item Anpassung 
\end{enumerate}

Wahrend der \textbf{Vorbereitung Phase} wird die Infrastruktur gebaut. Der Markteintritt soll mit schon funktionierte Dienstleistung erfolgen. Zu dieser Phase zahlt noch die Marketingvorbereitungsmaßnahmen. Dieser Phase befasst:
\begin{itemize}
\item Gestaltung der Suchmaschine
\item Anmeldung der Domain für Webauftritt
\item Gestaltung der Markenrichtlinien (Logo, Farben etc.)
\item Werbematerialvorbereitung
\item Gestaltung der Webauftritt, Mobile App und RSS Feed 
\item Testen der alle Funktionen 
\item Gestaltung der Infrastruktur (Büro, Miete der Rechnerkapazitäten oder Einkauf, Unternehmensrundung, Buchhaltung, Juristische Beratung etc.)
\end{itemize}

Nach der Vorbereitung wird \textbf{Markteintritt} durchgeführt. Dieser Phase ist schwierig und verlangt viel Konzentration und Aufwand. Es wird ausgeführt:
\begin{itemize}
\item Veröffentlichung der Webauftritt 
\item Start der Kommunikationspolitikmaßnahmen und Search Engine Optimization (SEO)
\end{itemize}

Mit Hilfe der \textbf{Kontrolle} kann man alle Parameter, die für Funktionieren der Dienstleistung wichtig sind, kontrollieren und entsprechende Maßnahmen durchzuführen. Man soll die Parameter so zu wählen, dass sie die Wirklichkeit spiegeln. Eine von Parameter könnte sein:
\begin{itemize}
\item Besucheranzahl von Webauftritt 
\item Anzahl der neue Anmeldungen 
\item Anzahl der Suchanfragen
\item Statistik an genutzte Datenquellen
\item Variablenkosten 
\end{itemize}

Zusammen mit der ständige Kontrolle sollte die \textbf{Anpassung} gemacht werden. Alle Maßnahmen soll gut angepasst werden, um maximale Erfolg und Verbesserung der Dienstleistung zu verfolgen.


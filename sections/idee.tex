\section{Geschäftsidee}

Unter Punkt 2 können Sie ausführlicher auf Ihr Vorhaben eingehen. Folgende Fragen können als Leitfaden dienen: Wie schätzen Sie die Erfolgsfaktoren für Ihr Angebot ein? Welche Marktstrategie wollen Sie verfolgen? Auch Angaben zum geplanten Firmennamen, zum Standort, zur Rechtsform und zur Firmenstruktur gehören in diesen Abschnitt. Darüber hinaus sollten Sie an dieser Stelle Ihre Unternehmensziele für die nächsten drei bis fünf Jahre formulieren.



\subsection{Gründungsvorgeschichte}

- Urheber der Geschäftsidee und vorhandene Schutzrechte sowie Verknüpfung mit vorhergehenden Projekten
- Einbindung des Gründungsvorhabens in das Umfeld der Hochschule bzw. der Forschungseinrichtung 

\subsection{Know-how Träger}
- kurze Vorstellung des geplanten Gründungsteams und Aufgabenverteilung der beteiligten Personen
- Bedeutung des an der Hochschule bzw. der Forschungseinrichtung erworbenen Know-hows für das   
  Gründungsvorhaben
- vorhandene betriebswirtschaftliche bzw. unternehmerische Erfahrungen bzw. Ausbildungen
- Einbindung weiterer wissenschaftlicher Berater, Partner, Mentoren etc.
\subsection{Innovation}
- Beschreibung der Technologie- oder Produktinnovation bzw. der wissensbasierten innovativen Dienstleistung
- Entwicklungsansatz und derzeitiger Stand der Umsetzung (ggf. Labor- oder Funktionsmuster)

\subsection{Projektplanung}
- projektbezogener Arbeitsplan für den Förderzeitraum und Ausblick auf Aktivitäten bis zur Marktreife
\newpage



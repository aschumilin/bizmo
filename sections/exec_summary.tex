\section{Zusammenfassung}

SemLit ist ein präzise und zuverlässige Web-Dienstleistung, die für Informationssuche in Bereich Informatik entwickelt. Auf Basis von semantische Quellenindexierung erlaubt SemLit real-time Suche in Informationquellen, Informationsabfrageverwaltung und benachbarte Informationsquellensuche.\\
Die Zielgruppe ist vor allem Studenten und Wissenschaftler in Bereich der  Ingenieurwesen und Technologie, Technische Hochschulen und Unternehmen.
Das Projekt wurde von zwei Studenten von Karlsruher Institut für Technologie gegründet. Das Team hast entsprechende Kenntnisse in Bereichen von Web Technologien, Semantic Technologien, Softwareentwicklung sowie Unternehemens- und Finanzverwaltung. \\
Auf der Basis von Finanzplanung und Entwicklungsstrategie lasst sich das Break-Even Punkt am 17. Monat der Unternehmenstätigkeit erwarten. Die Entwicklung bis Break-Even dient dafür eine gut funktionierte Infrastruktur zubauen, die als Grundstein der Erfolg dient. Die Chancen und Risiken Analyse zeigt, dass die Finanzsituation nach Break-Evan eine gute Aufsicht auf weitere erfolgreiche Entwicklung hat. Die geplante Umsatz nach 24 Monate umfasst ca. 40 000 Euro pro Monat.\\
Die genutzte Technologie könnte leicht am verwandte Bereiche eingesetzt werden, wie zum Beispiel Informationssuche und Verwaltung in wirtschaftliche und juristische Bereichen. Das beschriebene Einsatz kann sehr leicht durchgeführt werden, dank schon existierende Infrastruktur und Technologie.\\
SemLit hat sehr gute Aufsicht auf Markterfolg und könnte als effektive Investition berücksichtigen werden. 

